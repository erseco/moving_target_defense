\chapter{Metodología}

En este capítulo vamos a definir y a detallar las herramientas necesarias y los procesos que vamos a seguir para solucionar el problema planteado en base a los objetivos que presentamos en el segundo capítulo y a los antecedentes vistos en el tercero.

\section{Herramientas utilizadas}

\subsection{OWASP ZAP}

Analizar la seguridad es una tarea compleja que requiere tener en cuenta multitud de factores por lo que realizar dicho análisis de forma manual resulta poco menos que imposible. Por ello vamos a utilizar \texttt{ZAP} que es un analizador de vulnerabilidades de código abierto desarrollado por la organización \texttt{OWASP}. Dicho analizador fue una de las herramientas que obtuvo el premio \textit{Bossie 2015} al mejor software de red y seguridad de código abierto \cite{staff_bossie_2015}.

\bigskip
Aunque ZAP tiene una interfaz gráfica (ver \ref{fig:owasp_zap}) para nuestro proyecto vamos a automatizar su uso mediante la API disponible en lenguaje Python.


\begin{figure}[H]
\centering
\includegraphics[width=1.0\textwidth]{../images/owasp-zap-main-window}
\caption{Ventana principal de OWASP ZAP}
\label{fig:owasp_zap}
\end{figure}


\subsection{Docker}

Para simular diferentes servidores de una forma rápida se ha optado por usar un sistema de virtualización ligera basada en contenedores. Para esto se ha usado \textfb{Docker} que es uno de los sistemas de virtualización ligera basada en contenedores más utilizada. Para la orquestación de los diferentes contenedores (Servidor y Analizador) se utilizará el estándar \textbf{docker-compose} que nos permite definir un conjunto de contenedores de una forma sencilla en un fichero de texto YML.

\bigskip
La orquestación estará compuesta de dos contenedores, uno basado en \textit{Alpine Linux} en el que instalaremos el servidor que vayamos a analizar y un segundo contenedor basado en la imagen oficial de ZAP que podemos encontrar en la URL: \url{https://hub.docker.com/r/owasp/zap2docker-bare/}.

\subsection{Python}

Se ha decidido desarrollar este proyecto en el lenguaje de programación Python debido a estar familiarizado con este lenguaje y a la disponibilidad de una API de OWASP ZAP en dicho lenguaje. Además, aunque este proyecto no requiere un elevado rendimiento, diversas publicaciones indican un muy buen desempeño del lenguaje Python a la hora de trabajar con algoritmos genéticos \cite{merelo-guervos_comparison_2016}.

\section{Análisis general del problema}

Teniendo en cuenta la ingente cantidad de servicios de red con sus múltiples opciones de configuración se ha optado por limitar este proyecto a alterar y optimizar la configuración de un servidor HTTP, concretamente NGINX ya en los últimos años ha desbancado a Apache como el servidor HTTP  más utilizado del mundo\cite{w3techs_usage_2019}.

\bigskip
La configuración de NGINX, al igual que la de muchos otros servidores, se realiza mediante ficheros de texto plano con una sintaxis específica, la misma se puede encontrar definida en la siguiente URL: \url{http://nginx.org/en/docs/beginners_guide.html#conf_structure} y consta de múltiples directivas, podemos ver un ejemplo aquí: \ref{lst:nginx_config}.

\begin{lstlisting}[label={lst:nginx_config},caption={Ejemplo de fichero de configuración de NGINX}]
user nginx;
pid /var/run/nginx.pid;
worker_processes auto;

error_log /var/log/nginx/error.log warn;
events {
    worker_connections 1024;
}
http {
    include /etc/nginx/mime.types;
    default_type application/octet-stream;
    access_log /var/log/nginx/access.log;
    sendfile on;
    keepalive_timeout 65;
    gzip on;

    server {
        listen       80;
        server_name  example.com;

        access_log  /var/log/nginx/host.access.log  main;

        location / {

            add_header X-Frame-Options "SAMEORIGIN";
            proxy_pass http://juice-shop:3000;
            root   /usr/share/nginx/html;
            index  index.html index.htm;
        }

        # redirect server error pages to the static page /50x.html
        #
        error_page   500 502 503 504  /50x.html;
        location = /50x.html {
            root   /usr/share/nginx/html;
        }

    }
}
\end{lstlisting}

Como podemos observar un fichero de configuración de NGINX tiene diferentes directivas que se pueden configurar de diferentes maneras, tanto para optimizar como para simplemente simular una configuración distinta y así hacer creer a un atacante que hemos cambiado de servidor HTTTP o incluso de servidor físico.

\section{Banco de pruebas}

Para empezar definimos un banco de pruebas para comprobar que la herramienta OWASP ZAP servía para nuestro propósito. Para ello se modificó el código de ejemplo que proporcionan para su API en Python para hacerla funcionar en un entorno de contenedores. El script (zap.py \ref{zap.py}) se conecta a una determinada URL y ejecuta una serie de comprobaciones para terminar devolviendo el número de vulnerabilidades encontradas. Dichas vulnerabilidades están basadas en la escala CVSS por lo que hay vulnerabilidades consideradas críticas y otras que son solo sugerencias.

\bigskip
A la hora de obtener una web para testear se estuvo barajando el uso de diferentes entornos web como pueden ser Galileo (Perl), WordPress (PHP), una simple web realizada en html\ref{lst:simple_html_web} e incluso una aplicación web premeditadamente vulnerable desarrollada por OWASP llamada `Juice-Shop' realizada con \textbf{Node.js} pero al final se optó por utilizar el mensaje de bienvenida de `NGINX' (ver \ref{lst:nginx_welcome_message}). La configuración estándar de NGINX con dicho mensaje nos mostraba 5 alertas \ref{lst:owas_zap_welcome_message_alerts} siendo mucho mas fácil de gestionar que las 71 que obteníamos con `Juice-Shop'.

\bigskip
Una parte muy importante para el correcto funcionamiento de OWASP ZAP fue simular un dominio, en este caso utilizamos el conocido \textfb{example.com} que es un nombre reservado por la IANA (Internet Assigned Numbers Authority) en el RFC2606 (\cite{eastlake_reserved_1999}). Para hacer uso del mismo se ha optado por darle ese nombre de \textfb{host} al contenedor que levanta NGINX.


\begin{lstlisting}[language=html, label={lst:simple_html_web},caption={Web sencilla realizada en HTML puro}]
<!DOCTYPE html>
<html lang="es-ES">
  <head>
    <meta charset="utf-8">
    <title>Ejemplo de 2 párrafos</title>
  </head>
  <body>
    <p>Esto es un párrafo.</p>
    <p>Esto es otro párrafo.</p>
  </body>
</html>
\end{lstlisting}

\begin{lstlisting}[language=html, label={lst:nginx_welcome_message},caption={Mensaje de bienvenida de NGINX}]
# curl -v www.example.com
*   Trying 192.168.208.4:80...
* TCP_NODELAY set
* Connected to www.example.com (192.168.208.4) port 80 (#0)
> GET / HTTP/1.1
> Host: www.example.com
> User-Agent: curl/7.65.1
> Accept: */*
>
* Mark bundle as not supporting multiuse
< HTTP/1.1 200 OK
< Server: NGINX/1.17.2
< Date: Mon, 05 Aug 2019 13:42:32 GMT
< Content-Type: text/html
< Content-Length: 612
< Last-Modified: Tue, 23 Jul 2019 13:01:30 GMT
< Connection: keep-alive
< ETag: "5d37052a-264"
< Accept-Ranges: bytes
<
<!DOCTYPE html>
<html>
<head>
<title>Welcome to NGINX!</title>
<style>
    body {
        width: 35em;
        margin: 0 auto;
        font-family: Tahoma, Verdana, Arial, sans-serif;
    }
</style>
</head>
<body>
<h1>Welcome to NGINX!</h1>
<p>If you see this page, the NGINX web server is successfully installed and
working. Further configuration is required.</p>

<p>For online documentation and support please refer to
<a href="http://NGINX.org/">NGINX.org</a>.<br/>
Commercial support is available at
<a href="http://NGINX.com/">NGINX.com</a>.</p>

<p><em>Thank you for using NGINX.</em></p>
</body>
</html>
* Connection #0 to host www.example.com left intact
\end{lstlisting}

\begin{lstlisting}[language=json,label={lst:owas_zap_welcome_message_alerts},caption={Alerts showed with NGINX default configuration}]
[{'alert': 'Web Browser XSS Protection Not Enabled',
  'attack': '',
  'confidence': 'Medium',
  'cweid': '933',
  'description': 'Web Browser XSS Protection is not enabled, or is disabled by '
                 "the configuration of the 'X-XSS-Protection' HTTP response "
                 'header on the web server',
  'evidence': '',
  'id': '0',
  'messageId': '1',
  'method': 'GET',
  'name': 'Web Browser XSS Protection Not Enabled',
  'other': 'The X-XSS-Protection HTTP response header allows the web server to '
           "enable or disable the web browser's XSS protection mechanism. The "
           'following values would attempt to enable it: \n'
           'X-XSS-Protection: 1; mode=block\n'
           'X-XSS-Protection: 1; report=http://www.example.com/xss\n'
           'The following values would disable it:\n'
           'X-XSS-Protection: 0\n'
           'The X-XSS-Protection HTTP response header is currently supported '
           'on Internet Explorer, Chrome and Safari (WebKit).\n'
           'Note that this alert is only raised if the response body could '
           'potentially contain an XSS payload (with a text-based content '
           'type, with a non-zero length).',
  'param': 'X-XSS-Protection',
  'pluginId': '10016',
  'reference': 'https://www.owasp.org/index.php/XSS_(Cross_Site_Scripting)_Prevention_Cheat_Sheet\n'
               'https://www.veracode.com/blog/2014/03/guidelines-for-setting-security-headers/',
  'risk': 'Low',
  'solution': "Ensure that the web browser's XSS filter is enabled, by setting "
              "the X-XSS-Protection HTTP response header to '1'.",
  'sourceid': '3',
  'url': 'http://www.example.com',
  'wascid': '14'},
 {'alert': 'X-Content-Type-Options Header Missing',
  'attack': '',
  'confidence': 'Medium',
  'cweid': '16',
  'description': 'The Anti-MIME-Sniffing header X-Content-Type-Options was not '
                 "set to 'nosniff'. This allows older versions of Internet "
                 'Explorer and Chrome to perform MIME-sniffing on the response '
                 'body, potentially causing the response body to be '
                 'interpreted and displayed as a content type other than the '
                 'declared content type. Current (early 2014) and legacy '
                 'versions of Firefox will use the declared content type (if '
                 'one is set), rather than performing MIME-sniffing.',
  'evidence': '',
  'id': '1',
  'messageId': '1',
  'method': 'GET',
  'name': 'X-Content-Type-Options Header Missing',
  'other': 'This issue still applies to error type pages (401, 403, 500, etc) '
           'as those pages are often still affected by injection issues, in '
           'which case there is still concern for browsers sniffing pages away '
           'from their actual content type.\n'
           'At "High" threshold this scanner will not alert on client or '
           'server error responses.',
  'param': 'X-Content-Type-Options',
  'pluginId': '10021',
  'reference': 'http://msdn.microsoft.com/en-us/library/ie/gg622941%28v=vs.85%29.aspx\n'
               'https://www.owasp.org/index.php/List_of_useful_HTTP_headers',
  'risk': 'Low',
  'solution': 'Ensure that the application/web server sets the Content-Type '
              'header appropriately, and that it sets the '
              "X-Content-Type-Options header to 'nosniff' for all web pages.\n"
              'If possible, ensure that the end user uses a '
              'standards-compliant and modern web browser that does not '
              'perform MIME-sniffing at all, or that can be directed by the '
              'web application/web server to not perform MIME-sniffing.',
  'sourceid': '3',
  'url': 'http://www.example.com',
  'wascid': '15'},
 {'alert': 'X-Frame-Options Header Not Set',
  'attack': '',
  'confidence': 'Medium',
  'cweid': '16',
  'description': 'X-Frame-Options header is not included in the HTTP response '
                 "to protect against 'ClickJacking' attacks.",
  'evidence': '',
  'id': '2',
  'messageId': '1',
  'method': 'GET',
  'name': 'X-Frame-Options Header Not Set',
  'other': '',
  'param': 'X-Frame-Options',
  'pluginId': '10020',
  'reference': 'http://blogs.msdn.com/b/ieinternals/archive/2010/03/30/combating-clickjacking-with-x-frame-options.aspx',
  'risk': 'Medium',
  'solution': 'Most modern Web browsers support the X-Frame-Options HTTP '
              "header. Ensure it's set on all web pages returned by your site "
              '(if you expect the page to be framed only by pages on your '
              "server (e.g. it's part of a FRAMESET) then you'll want to use "
              'SAMEORIGIN, otherwise if you never expect the page to be '
              'framed, you should use DENY. ALLOW-FROM allows specific '
              'websites to frame the web page in supported web browsers).',
  'sourceid': '3',
  'url': 'http://www.example.com',
  'wascid': '15'},
 {'alert': 'Web Browser XSS Protection Not Enabled',
  'attack': '',
  'confidence': 'Medium',
  'cweid': '933',
  'description': 'Web Browser XSS Protection is not enabled, or is disabled by '
                 "the configuration of the 'X-XSS-Protection' HTTP response "
                 'header on the web server',
  'evidence': '',
  'id': '6',
  'messageId': '7',
  'method': 'GET',
  'name': 'Web Browser XSS Protection Not Enabled',
  'other': 'The X-XSS-Protection HTTP response header allows the web server to '
           "enable or disable the web browser's XSS protection mechanism. The "
           'following values would attempt to enable it: \n'
           'X-XSS-Protection: 1; mode=block\n'
           'X-XSS-Protection: 1; report=http://www.example.com/xss\n'
           'The following values would disable it:\n'
           'X-XSS-Protection: 0\n'
           'The X-XSS-Protection HTTP response header is currently supported '
           'on Internet Explorer, Chrome and Safari (WebKit).\n'
           'Note that this alert is only raised if the response body could '
           'potentially contain an XSS payload (with a text-based content '
           'type, with a non-zero length).',
  'param': 'X-XSS-Protection',
  'pluginId': '10016',
  'reference': 'https://www.owasp.org/index.php/XSS_(Cross_Site_Scripting)_Prevention_Cheat_Sheet\n'
               'https://www.veracode.com/blog/2014/03/guidelines-for-setting-security-headers/',
  'risk': 'Low',
  'solution': "Ensure that the web browser's XSS filter is enabled, by setting "
              "the X-XSS-Protection HTTP response header to '1'.",
  'sourceid': '3',
  'url': 'http://www.example.com/robots.txt',
  'wascid': '14'},
 {'alert': 'Web Browser XSS Protection Not Enabled',
  'attack': '',
  'confidence': 'Medium',
  'cweid': '933',
  'description': 'Web Browser XSS Protection is not enabled, or is disabled by '
                 "the configuration of the 'X-XSS-Protection' HTTP response "
                 'header on the web server',
  'evidence': '',
  'id': '7',
  'messageId': '8',
  'method': 'GET',
  'name': 'Web Browser XSS Protection Not Enabled',
  'other': 'The X-XSS-Protection HTTP response header allows the web server to '
           "enable or disable the web browser's XSS protection mechanism. The "
           'following values would attempt to enable it: \n'
           'X-XSS-Protection: 1; mode=block\n'
           'X-XSS-Protection: 1; report=http://www.example.com/xss\n'
           'The following values would disable it:\n'
           'X-XSS-Protection: 0\n'
           'The X-XSS-Protection HTTP response header is currently supported '
           'on Internet Explorer, Chrome and Safari (WebKit).\n'
           'Note that this alert is only raised if the response body could '
           'potentially contain an XSS payload (with a text-based content '
           'type, with a non-zero length).',
  'param': 'X-XSS-Protection',
  'pluginId': '10016',
  'reference': 'https://www.owasp.org/index.php/XSS_(Cross_Site_Scripting)_Prevention_Cheat_Sheet\n'
               'https://www.veracode.com/blog/2014/03/guidelines-for-setting-security-headers/',
  'risk': 'Low',
  'solution': "Ensure that the web browser's XSS filter is enabled, by setting "
              "the X-XSS-Protection HTTP response header to '1'.",
  'sourceid': '3',
  'url': 'http://www.example.com/sitemap.xml',
  'wascid': '14'}]
\end{lstlisting}

\bigskip
Una vez hemos generado nuestro laboratorio para pruebas y viendo que se NGINX se ejecuta correctamente pasamos a probar como va cambiando la respuestas de OWASP ZAP dependiendo de la configuración de nuestro NGINX. Para ello agregamos la cabecera \begin{verbatim}add_header X-Frame-Options "SAMEORIGIN";\end{verbatim} y ejecutamos el conjunto, dándonos 4 vulnerabilidades en lugar de las 5 anteriores. Esta prueba nos indica que efectivamente la modificación de los parámetros de NGINX puede darnos configuraciones más o menos seguras.

\section{Eligiendo los parámetros}

La última versión de NGINX (1.17.2) posee 739 directivas de configuración por lo que para simplificar nuestra tarea vamos a limitarnos a un conjunto mas pequeño de directivas (ver tabla \ref{table:apache_nginx_directives}).

\bigskip
En un primer momento pensamos en basarnos en las directivas \textfb{STIG}, pero estas directivas sólo están definidas para el servidor Apache. Por suerte al ser HTTP un estándar muchas de las directivas de Apache tienen su símil en NGINX por lo que hemos extraído un conjunto de las mas representativas y que además tuvieran su equivalente para el servidor NGINX. Las puntuaciones se basan en el sistema de puntuación CVSS de acuerdo con los resultados devueltos por ZAP con el código de prueba. También se han agregado algunas cabeceras identificativas como pueden ser la identificación del servidor (ver tabla \ref{table:ngingx_headers}) así como la directivas de compresión \texttt{gzip} para tener una mayor entropía.

\begin{table}[H]
\begin{tabular}{|l|l|l|l|}
\hline
Id & STIG ID & Configuración Apache  & Equivalente NGINX              \\ \hline
0  & V-13730 & MaxClients            & worker\_connections            \\ \hline
1  & V-13726 & KeepAliveTimeout      & keepalive\_timeout             \\ \hline
2  & V-13732 & FollowSymLinks        & disable\_symlinks              \\ \hline
3  & V-13735 & Indexes               & autoindex                      \\ \hline
4  & V-13724 & Timeout               & send\_timeout                  \\ \hline
5  & V-13738 & LimitRequestFieldsize & large\_client\_header\_buffers \\ \hline
6  & V-13736 & LimitRequestBody      & client\_max\_body\_size        \\ \hline
7  & V-6724  & ServerTokens          & server\_tokens                 \\ \hline
8  &         &                       & gzip                           \\ \hline
\end{tabular}
\label{table:apache_nginx_directives}
\caption{Lista de directivas NGINX utilizadas}
\end{table}

\begin{table}[H]
\begin{tabular}{|l|l|}
\hline
Id & Cabecera                       \\ \hline
9  & X-Frame-Options                \\ \hline
10  & X-Powered-By                   \\ \hline
11 & X-Content-Type-Options         \\ \hline
12 & Server                         \\ \hline
\end{tabular}
\label{table:ngingx_headers}
\caption{Lista de cabeceras HTTP utilizadas}
\end{table}


\bigskip
Pasamos a detallar que realiza cada directiva escogida y sus posibles valores:

\subsection{worker\_connections}

\begin{table}[H]
\begin{tabular}{|l|l|}
\hline
Sintaxis      & worker\_connections number; \\ \hline
Por defecto   & worker\_connections 512;     \\ \hline
Contexto      & events     \\ \hline
\end{tabular}
\end{table}

Establece el número máximo de conexiones simultáneas que pueden ser abiertas por un proceso de NGINX.

\bigskip
Debe tenerse en cuenta que este número incluye todas las conexiones (por ejemplo, conexiones con servidores proxy, entre otros), no sólo las conexiones con clientes. Otra consideración es que el número real de conexiones simultáneas no puede exceder el límite actual del número máximo de archivos abiertos.

\subsection{keepalive\_timeout}

\begin{table}[H]
\begin{tabular}{|l|l|}
\hline
Sintaxis      & keepalive\_timeout timeout; \\ \hline
Por defecto   & keepalive\_timeout 75s;     \\ \hline
Contexto      & http, server, location     \\ \hline
\end{tabular}
\end{table}

Establece un tiempo de espera durante el cual una conexión cliente permanecerá abierta en el lado del servidor. El valor cero deshabilita las conexiones de cliente Keep-alive. La cabecera `Keep-Alive: timeout=time' está soportada por Firefox y Chrome. Internet Explorer cierra las conexiones abiertas por sí mismo en unos 60 segundos.

\subsection{disable\_symlinks}

\begin{table}[H]
\begin{tabular}{|l|l|}
\hline
Sintaxis      & disable\_symlinks on \textbar  off; \\ \hline
Por defecto   & disable\_symlinks off;     \\ \hline
Contexto      & http, server, location     \\ \hline
\end{tabular}
\end{table}

Determina cómo se deben tratar los enlaces simbólicos al abrir archivos. Cuando está desactivado los enlaces simbólicos en la ruta de acceso están permitidos y no están marcados. Este es el comportamiento por defecto. Cuando está activado y algún componente de la ruta de acceso es un enlace simbólico, se deniega el acceso a ese archivo.

\subsection{autoindex}

\begin{table}[H]
\begin{tabular}{|l|l|}
\hline
Sintaxis      & autoindex on \textbar  off; \\ \hline
Por defecto   & autoindex off;     \\ \hline
Contexto      & http, server, location     \\ \hline
\end{tabular}
\end{table}

Cuando está activado muestra el contenido de los directorios, en caso contrario no muestra nada.

\subsection{send\_timeout}

\begin{table}[H]
\begin{tabular}{|l|l|}
\hline
Sintaxis      & send\_timeout time; \\ \hline
Por defecto   & send\_timeout 60s;     \\ \hline
Contexto      & http, server, location     \\ \hline
\end{tabular}
\end{table}

Establece el tiempo de espera para transmitir una respuesta al cliente. El tiempo de espera se establece sólo entre dos operaciones de escritura sucesivas, no para la transmisión de la respuesta completa. Si el cliente no recibe nada en este tiempo, la conexión se cierra.

\subsection{large\_client\_header\_buffers}

\begin{table}[H]
\begin{tabular}{|l|l|}
\hline
Sintaxis      & large\_client\_header\_buffers number size; \\ \hline
Por defecto   & large\_client\_header\_buffers 4 8k;     \\ \hline
Contexto      & http, server     \\ \hline
\end{tabular}
\end{table}

Establece el número máximo y el tamaño de los búferes utilizados para leer los encabezados de solicitudes de clientes grandes. Una línea de petición no puede exceder el tamaño de un búfer, o el error 414 (Request-URI Too Large) es devuelto al cliente. Un campo de encabezado de solicitud no puede exceder el tamaño de un búfer también, o el error 400 (Bad Request) es devuelto al cliente. Los búferes se asignan sólo bajo demanda. Por defecto, el tamaño del búfer es igual a 8K bytes. Si una vez finalizada la tramitación de la solicitud, la conexión pasa al estado de espera, se liberan estos búferes.

\subsection{client\_max\_body\_size}

\begin{table}[H]
\begin{tabular}{|l|l|}
\hline
Sintaxis      & client\_max\_body\_size size; \\ \hline
Por defecto   & client\_max\_body\_size 1m;     \\ \hline
Contexto      & http, server, location     \\ \hline
\end{tabular}
\end{table}

Establece el tamaño máximo permitido del cuerpo de la solicitud del cliente, especificado en el campo `Content-Length' del encabezado de la solicitud. Si el tamaño de una solicitud excede el valor configurado, el error 413 (Request Entity Too Large) se devuelve al cliente. Tenga en cuenta que los navegadores no pueden mostrar correctamente este error. Configurar el tamaño a 0 desactiva la comprobación del tamaño del cuerpo de la solicitud del cliente.

\subsection{server\_tokens}

\begin{table}[H]
\begin{tabular}{|l|l|}
\hline
Sintaxis      & server\_tokens on \textbar  off; \\ \hline
Por defecto   & server\_tokens on;     \\ \hline
Contexto      & http, server, location     \\ \hline
\end{tabular}
\end{table}

Habilita o deshabilita la emisión de la versión NGINX en las páginas de error y en el campo `Server' del encabezado de respuesta.

\subsection{gzip}

\begin{table}[H]
\begin{tabular}{|l|l|}
\hline
Sintaxis      & gzip on \textbar  off; \\ \hline
Por defecto   & gzip off;     \\ \hline
Contexto      & http, server, location     \\ \hline
\end{tabular}
\end{table}

Habilita o deshabilita la compresión de las respuestas HTTP.

\subsection{X-Frame-Options}

\begin{table}[H]
\begin{tabular}{|l|l|}
\hline
Sintaxis      & X-Frame-Options: DENY \textbar  SAMEORIGIN \textbar  ALLOW-FROM url; \\ \hline
Contexto      & server, location     \\ \hline
\end{tabular}
\end{table}

La cabecera `X-Frame-Options' puede ser usada para indicar si debería permitírsele a un navegador renderizar una página de forma embebida . Las páginas web pueden usarlo para evitar ataques de \textit{clickjacking}, asegurándose que su contenido no es embebido en otros sitios.

\subsection{X-Powered-By}

\begin{table}[H]
\begin{tabular}{|l|l|}
\hline
Sintaxis      & X-Powered-by: NGINX; \\ \hline
Contexto      & server, location     \\ \hline
\end{tabular}
\end{table}

La cabecera `X-Powered-By' se usa para especificar con que software se ha generado la respuesta por parte del servidor.

\bigskip
Se recomienda no dar información demasiado extensa en dicha cabecera ya que puede revelar detalles que pueden facilitar la tarea de encontrar y explotar fallos de seguridad.


\subsection{X-Content-Type-Options}

\begin{table}[H]
\begin{tabular}{|l|l|}
\hline
Sintaxis      & X-Content-Type-Options: nosniff; \\ \hline
Contexto      & server, location     \\ \hline
\end{tabular}
\end{table}

El encabezado HTTP de respuesta `X-Content-Type-Options' es un marcador utilizado por el servidor para indicar que los tipos \textit{MIME} anunciados en los encabezados `Content-Type' no se deben cambiar ni seguir. Esto permite desactivar el `MIME type sniffing'.

\bigskip
Introducido por Microsoft en Internet Explorer 8 e implementado paulatinamente por el resto de navegadores, ayuda a los administradores puedan bloquear el rastreo de contenido, pudiendo transformar tipos MIME no ejecutables en tipos MIME ejecutables.

\subsection{server}

\begin{table}[H]
\begin{tabular}{|l|l|}
\hline
Sintaxis      & Server: NGINX 1.12; \\ \hline
Contexto      & server, location     \\ \hline
\end{tabular}
\end{table}

La cabecera `Server' contiene la información acerca del software usado por el servidor.

\bigskip
Al igual que con la cabecera `X-Powered-By', se recomienda no dar información demasiado extensa en dicha cabecera ya que puede revelar detalles que pueden facilitar la tarea de encontrar y explotar fallos de seguridad.


\section{Generación de configuraciones}

\bigskip
Para generar las distintas configuraciones de NGINX utilizamos la librería \textfb{nginx-config-builder} desarrollada por Linkedin y licenciada bajo la BSD. Dicha librería la podemos encontrar en \url{https://github.com/linkedin/nginx-config-builder}. Haciendo uso de esta librería desarrollamos un script \ref{generate_nginx_config.py} que en base a un cromosoma generaba una configuración, sin saber todavía si dicha configuración podía funcionar o no.

\bigskip
El siguiente paso es conseguir saber primeramente si la configuración funciona y después saber si la misma configuración funcional es segura. Para lo primero el propio NGINX tiene una herramienta de comprobación de configuración, la podemos invocar con el comando \textfb{NGINX -t <archivo-configuracion.conf>}, dicha herramienta nos comprueba que la sintaxis del fichero de configuración sea correcto y NGINX pueda ejecutarse sin problemas. Para lo segundo configuraremos un servidor NGINX con dicha configuracion y haremos uso de la herramienta ZAP, para poder hacerlo de forma dinámica y sencilla utilizaremos la potencia de Docker.

\bigskip
Nuestra función de evaluación utilizará una combinación de estas dos herramientas, ya que el primero nos servirá para descartar directamente las configuraciones erróneas y la segunda nos permitirá establecer la calidad de una configuración con un valor escalar.

\section{Implementación del algoritmo genético}

La implementación del algoritmo genético utiliza una combinación de procesos de selección, cruzamiento y mutación para mejorar la solución a través de varias generaciones.
\bigskip

Aunque en un primero momento se planteó el uso de algún `framework' de algoritmos genéticos como puede ser \textfb{DEAP}, desarrollado en Python, pero se optó en su lugar por utilizar un algoritmo sencillo que cubría de sobra nuestras necesidades.

\bigskip
Para la realización del algoritmo genético hemos utilizado como base uno de los múltiples ejemplos que hay en Internet. Concretamente el del estupendo tutorial que podemos encontrar en la web \textfb{robologs.net} concretamente en esta URL: \url{https://robologs.net/2015/09/01/}.

\bigskip
El sistema está diseñado para lograr el objetivo del planteamiento del problema. En este diseño, se han desarrollado dos algoritmos y en total se han utilizado cuatro guiones en conjunto para lograr el objetivo. Para los experimentos se han utilizado una serie de herramientas. El algoritmo genético responderá a la parte principal del enunciado del problema "Cómo reforzar las vulnerabilidades de seguridad causadas por la mala configuración humana o por una cadena de configuración inadecuada". El componente y las herramientas se discuten en este capítulo. El segundo algoritmo es puntuar la idoneidad de las soluciones.

\bigskip
El algoritmo genético se encarga de llamar a los anteriores scripts para generar soluciones de seguridad. En primer lugar, inicializa las soluciones aleatorias, que luego pasan por los procesos de selección, que se describen en detalle más adelante en este capítulo. El algoritmo genético depende de la puntuación de aptitud de cada solución para ir más allá en el proceso de selección, lo que significa que es responsabilidad del algoritmo de puntuación de aptitud dar puntuaciones para las soluciones y enviarlas más allá del algoritmo genético.

\bigskip
El algoritmo de puntuación (fitness.py \ref{fitness.py}) será el responsable de proporcionar al algoritmo genético las puntuaciones de fitness de las soluciones de seguridad. El sistema de puntuación se basa en las preferencias previas de STIG, que proporciona las vulnerabilidades, que será devuelto como un valor escalar dado por ZAP.
