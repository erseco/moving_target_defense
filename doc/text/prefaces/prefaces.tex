\begin{center}
{\LARGE\bfseries\titulo}\\
\end{center}
\begin{center}
\autor\
\end{center}

\section*{Resumen}

\bigskip
\noindent{\textbf{Palabras clave}: \textit{\keywords}\\

Además de realizar una labor determinada de forma eficiente, los servicios informáticos deben ser capaces de evitar los ataques y de detectar los que haya. Una técnica de defensa consiste en convertirse en un objetivo móvil, que varíe el perfil de forma que los atacantes no lo reconozcan.

\bigskip
Mediante algoritmos evolutivos trataremos de configurar diferentes servicios de forma que se maximice la diversidad, a la vez que se optimice la seguridad y las prestaciones.

\newpage
\begin{center}
{\LARGE\bfseries\tituloEng}\\
\end{center}
\begin{center}
\autor\
\end{center}

\section*{Extended abstract}

\bigskip
\noindent{\textbf{Keywords}: \textit{\keywordsen}.\\

Cyber attacks are one of the biggest problems for many organizations. That organizations are investing so many resources in detecting cyber-attacks, but they still have serious difficulties to prevent them.

\bigskip
The common way to perform an attack is using a list of known vulnerabilities. Many of these vulnerabilities can be caused by a bad configuration.

\bigskip
A simple protection against computer security threats can be implemented by properly tuning existing system software without the needed of expensive security solutions.

\bigskip
This project shows a low-cost technique to prevent computer attacks. This technique consists of constantly modifying the configuration of a server.

\bigskip
Using a genetic algorithm, many different possible configurations are mutated to find an optimal solution. This optimal configuration is regularly applied, so the systems information previously collected by a potential attacker is no longer effective, achieving a simple protection layer while optimizing the service configuration.


\newpage
\thispagestyle{empty}
\
\vspace{10cm}

\noindent\rule[-1ex]{\textwidth}{2pt}\\[4.5ex]

% \section*{Declaración de Originalidad del TFM}

Yo, \textbf{\autor}, alumno de la titulación \textbf{\master} de la \textbf{\escuela\ de la \universidad}, declaro que el presente Trabajo de Fin de Máster es original, no habiéndose utilizado fuentes sin ser citadas debidamente. De no cumplir con este compromiso, soy consciente de que, de acuerdo con la Normativa de Evaluación y de Calificación de los estudiantes de la Universidad de Granada de 20 de mayo de 2013, \textit{esto conllevará automáticamente la calificación numérica de cero [...] independientemente del resto de las calificaciones que el estudiante hubiera obtenido. Esta consecuencia debe entenderse sin perjuicio de las responsabilidades disciplinarias en las que pudieran incurrir los estudiantes que plagien.}

\bigskip
Asimismo, autorizo la ubicación de la siguiente copia de mi Trabajo de Fin de Máster (\textit{\titulo}) en la biblioteca del centro para que pueda ser consultada por las personas que lo deseen.

\bigskip
Además, este mismo trabajo está publicado bajo la licencia \textbf{Creative Commons Attribution-ShareAlike 4.0} \cite{CC}, dando permiso para copiarlo y redistribuirlo en cualquier medio o formato, también de adaptarlo de la forma que se quiera, pero todo esto siempre y cuando se reconozca la autoría y se distribuya con la misma licencia que el trabajo original. Todo el código fuente así como este documento en formato {\tt LaTeX} se puede encontrar en el siguiente repositorios de {\tt GitHub}: \url{https://github.com/erseco/moving_target_defense}.

\bigskip
Y para que así conste firmo el presente documento.

\vspace{3cm}

\noindent Fdo: \autor

\vspace{3cm}

\begin{flushright}
\ciudad, a \today
\end{flushright}

\newpage
\thispagestyle{empty}
\
\vspace{3cm}

\noindent\rule[-1ex]{\textwidth}{2pt}\\[4.5ex]

D. \textbf{\tutor}, profesor del \textbf{Departamento de Arquitectura y Tecnología de los Computadores} de la \textbf{\universidad}.

\vspace{0.5cm}

\vspace{0.5cm}

\textbf{Informa:}

\vspace{0.5cm}

Que el presente trabajo, titulado \textit{\textbf{\titulo}}, ha sido realizado bajo su supervisión por \textbf{\autor}, y
autoriza la defensa de dicho trabajo ante el tribunal que corresponda.

\vspace{0.5cm}

Y para que conste, expide y firma el presente informe en \ciudad\ a \today.

\vspace{1cm}

\textbf{El tutor:}

\vspace{3cm}

%\begin{figure}[H]
%\includegraphics[width=0.3\textwidth]{../../firmaJJ}
%\end{figure}

\noindent \textbf{\tutor}

\chapter*{Agradecimientos}
\thispagestyle{empty}

\vspace{1cm}

A Georgia, que algún día será mejor ingeniera que su tío.

\bigskip
Al ZX Spectrum 128K de mis hermanos, porque sin él no habría llegado hasta aquí.
